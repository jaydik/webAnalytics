\documentclass[11pt]{beamer}
\usetheme{PaloAlto}
\usepackage[utf8]{inputenc}
\usepackage{amsmath}
\usepackage{amsfonts}
\usepackage{amssymb}
\author{Jon Dickerson}
\title{News Category Prediction}
%\setbeamercovered{transparent} 
%\setbeamertemplate{navigation symbols}{} 
%\logo{} 
\institute{Stevens Institute of Technology} 
\date{2016-12-06} 
\subject{Web Analytics} 
\begin{document}

\begin{frame}
\titlepage
\end{frame}

\begin{frame}
\tableofcontents
\end{frame}

\section{Introduction}
\begin{frame}{Introduction}
	The dataset I chose to analyze was Kaggle's News Aggregator Dataset.\footnote{https://www.kaggle.com/uciml/news-aggregator-dataset} It contains "headlines and categories of 400k news stories from 2014". The goal of this project is to attempt to predict the category of a news story given its headline.
\end{frame}

\section{Data Structure}
\begin{frame}{Data Structure}
	The columns included in this dataset are:
	\begin{itemize}
		\item[-] ID : the numeric ID of the article
		\item[-] TITLE : the headline of the article
		\item[-] URL : the URL of the article
		\item[-] PUBLISHER : the publisher of the article
		\item[-] CATEGORY : the category of the news item; one of:
			\begin{itemize}
				\item[-] b: business
				\item[-] t: science and technology
				\item[-] e: entertainment
				\item[-] m: health
			\end{itemize}
		\item[-] STORY : alphanumeric ID of the news story that the article discusses
		\item[-] HOSTNAME : hostname where the article was posted
		\item[-] TIMESTAMP : approximate timestamp of the article's publication
	\end{itemize}
\end{frame}

\section{Code}
\begin{frame}{Code Plan}
	The code consists of a few small scripts for ease of use:
	\begin{itemize}
		\item[-] \textit{00-feature\_extraction.py}: stem and vectorize the headlines
		\item[-] \textit{01-build\_models.py}: train a few models to compare performance
		\item[-] \textit{02-evaluate\_models.py}: evaluate the above models
		\item[-] \textit{03-tune\_winner.py}: take the model with the best baseline performance and tune it using grid search
		\item[-] \textit{04-evaluate\_winner.py}: rerun the evaluation on the tuned model
	\end{itemize}
\end{frame}

\section{Results}
\begin{frame}{Results}
	The winning model was a linear support vector machine with gradient descent. The final confusion matrix is given below.
\[
	\begin{bmatrix}
		26,832 & 486 & 231 & 1535 \\ 
		453 & 37,170 & 120 & 296 \\
		475 & 322 & 10,543 & 156 \\
		1430 & 489 & 126 & 24,941
	\end{bmatrix}
\]
\end{frame}

\end{document}
